\documentclass{article}

\title{The feeling of the age: A quantitative analysis of the correlation between novelistic and economic sentiment}
\author{Laura G\'ati, Daniella G\'ati}
\date{\today}

\begin{document}

\maketitle

\begin{abstract}
	This paper introduces a novel variable into the analysis of economic fluctuations: literary sentiment. We construct our literary sentiment variable by using text analysis methods on award-winning novels in the years 1948-2018, assigning them quantitative sentiment scores. We then use a structural VAR framework to correlate these data with a range of economic variables, aimed at capturing both the stance of the business cycle (e.g. VIX, stock prices, GDP and hours worked), as well as the general long-run economic outlook (e.g. the natural rate of interest and long-run survey expectations). We then back out the effects of structural innovations to literary sentiment in our short-run and long-run VAR specifications. Thus, our analysis isolates the effect of ``the feeling of the age'' as expressed through award-winning novels on short-run and long-run economic outcomes. 
	
	We hypothesize that while novels may not function as ex ante predictors of economic developments due to the time it takes to write and publish them, they do become valuable summary statistics ex post of the structural conditions of the economy at a particular time, as well as of medium-run optimism/pessimism regarding the economy. 
	%Our findings indicate….
\end{abstract}

\end{document}